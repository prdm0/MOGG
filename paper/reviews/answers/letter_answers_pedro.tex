\documentclass[version=last,12pt,{"maintainersDelight"}]{scrlttr2}

% For Debuging the Layout  ----------------------------------------------------
% \LoadLetterOption{visualize}
% \showfields{head,address,location,refline,foot}
% \setkomafont{field}{\color{blue}}
% -----------------------------------------------------------------------------

\usepackage{mathpazo}
\usepackage{amssymb,amsmath}
\usepackage{ifxetex,ifluatex}
\usepackage{fixltx2e} % provides \textsubscript
\ifnum 0\ifxetex 1\fi\ifluatex 1\fi=0 % if pdftex
  \usepackage[T1]{fontenc}
  \usepackage[utf8]{inputenc}
\else % if luatex or xelatex
  \ifxetex
    \usepackage{mathspec}
  \else
    \usepackage{fontspec}
  \fi
  \defaultfontfeatures{Ligatures=TeX,Scale=MatchLowercase}
\fi
% use upquote if available, for straight quotes in verbatim environments
\IfFileExists{upquote.sty}{\usepackage{upquote}}{}
% use microtype if available
\IfFileExists{microtype.sty}{%
\usepackage{microtype}
\UseMicrotypeSet[protrusion]{basicmath} % disable protrusion for tt fonts
}{}
% \usepackage[margin=1in]{geometry} interfers with the areaset in my lco_default
%\usepackage[margin=1in]{geometry}
\usepackage[unicode=true,pdfversion=1.6]{hyperref} % PrintScaling needs 1.6
\PassOptionsToPackage{usenames,dvipsnames}{color} % color is loaded by hyperref
\hypersetup{
            pdfauthor={Prof.~Dr.~Pedro Rafael D. Marinho},
            colorlinks=true,
            linkcolor=red,
            citecolor=Blue,
            urlcolor=blue,
            breaklinks=true,
            pdfprintscaling=None} % signal the PDF viewer "no auto scale", PDF >= 1.6
\urlstyle{same}  % don't use monospace font for urls
% make links footnotes instead of hotlinks (default)
\renewcommand{\href}[2]{#2\footnote{\url{#1}}}
\setlength{\emergencystretch}{3em}  % prevent overfull lines
\providecommand{\tightlist}{%
  \setlength{\itemsep}{0pt}\setlength{\parskip}{0pt}}
\setcounter{secnumdepth}{0}
% Redefines (sub)paragraphs to behave more like sections
\ifx\paragraph\undefined\else
\let\oldparagraph\paragraph
\renewcommand{\paragraph}[1]{\oldparagraph{#1}\mbox{}}
\fi
\ifx\subparagraph\undefined\else
\let\oldsubparagraph\subparagraph
\renewcommand{\subparagraph}[1]{\oldsubparagraph{#1}\mbox{}}
\fi

\makeatletter
\usepackage{graphicx}
% grffile has become a legacy package: https://ctan.org/pkg/grffile
\IfFileExists{grffile.sty}{%
\usepackage{grffile}
}{}
% set default figure placement to htbp
\def\fps@figure{htbp}
\setkomavar{signature}{Prof.~Dr.~Pedro Rafael D. Marinho and co-authors}
\makeatother



\KOMAoptions{fromemail=true}
\setkomavar{fromemail}{pedro.rafael.marinho@gmail.com}
\KOMAoptions{fromurl=true}
\setkomavar{fromurl}{https://prdm0.rbind.io/}



\setkomavar{yourmail}{pedro.rafael.marinho@gmail.com}




\setkomavar{subject}{Marshall and Olkin-G and Gamma-G family of
distribution: properties and applications}
\setkomavar{fromname}{Response letter from the authors}
\setkomavar{fromaddress}{Department of Statistics\\Federal University of
Paraíba\\João Pessoa, Paraíba - PB, Brazil}


\begin{document}


\begin{letter}{\textbf{Replies to reviewers}\\\textbf{REVSTAT}\\}
\opening{Dear \textbf{REVSTAT} Executive Editor and Reviews,}

in this letter, I will respond to each of the reviewers valuable
contributions. We try to accept all suggestions, aiming at a significant
improvement of our article. After each questioning, the respective
answers will be given. As suggested by the second reviewer, the title of
the paper was modified and better represents the proposal of the work.

\textbf{REVIEW I}:

1 -- The equality \(h(x)=f(x)/(1-F(x))\) is a well known relation
between hazard, density and survival functions. It is not a definition of density or risk functions. Therefore, in page 3, line 12, the part of the sentence ``defined by \(h(x)=f(x)/(1-F(x))\)'' should be omitted.

\textcolor{red}{Thank you. Done}. 

\emph{Response}:

2 -- The expression (3.1) in page 6 corresponds to the log-likelihood of the MO-\(\Gamma\)-\(G\) distribution, 
thus considering a Weibull distribution
for \(G\) (although there is an error in the last sum 
of line 1), not a general distribution with a \(\eta\) 
vector of parameters. Therefore, the text or the expression 
should be written accordingly.

\emph{Response}:

3 -- It is not clear why it is considered the Weibull distribution (for \(G\)) in Figure 1, eventually in expression (3.1) and in 
the applications, whereas for the simulations it is chosen the 
Dagum distribution. These choices should be clarified.

\emph{Response}:
\textcolor{red}{Indeed, the way in which the application section was introduced suggests that an attempt was made to fit the $G\sim Weibull$ model as a suitable distribution to the current datasets. In order to remove some ambiguity, we opted to extend the simulation studies, where we now include the $G\sim Weibull$ in the simulations.}

4 -- As the Dagum distribution is used in the simulations, the generated PDF should also be in Table 1.

\emph{Response}:
\textcolor{red}{Thank you for pointing that out. The Dagum distribution was now included in Table 1.}

5 -- As the \texttt{goodness.fit} function gives several statistics, the authors should justify why they only chose the Anderson Darling and Cram\'er-von Mises statistics.

\emph{Response}:
\textcolor{red}{The use of the Anderson-Darling  $A^*$ and Cram\'er-von Mises $W^*$  statistics, corrected by Chen and Balakrishnan, in the paper entitled ``A General Purpose Approximate Goodness-of-Fit Test'' (see   \url{https://www.tandfonline.com/doi/abs/10.1080/00224065.1995.11979578}) is indicated for comparing mo\-dels that are not necessarily embedded. Hence, considering the statistics returned by the \texttt{goodness.fit} function and the distributions compared, we understand that these statistics are the most suitable ones for 
the comparisons.}

\textcolor{red}{The \textbf{AdequacyModel} package was developed 
by one of the authors of this paper and it was published at   \url{https://journals .plos.org/plosone/article?id=10.1371/journal.pone.0221487}. 
A code inspection of the \texttt{goodness.fit} function can be done by looking at the link   \href{https://github.com/prdm0/AdequacyModel/blob/master/R/goodness.fit.R}{link} on GitHub, where it is hosted the development versioning of the package.}

\emph{Minor points:}

1 -- page 1, line 8 of Section 1, correct for ``vector''.

\emph{Response}: \textcolor{red}{Done.}

2 -- page 2: confirm the expressions of \(Q_{MO-G}(u)\) (line 9) and
\(W\) (line22).

\emph{Response}: \textcolor{red}{Done.}

3 -- page 2, line -4 of Section 1, correct for ``are''.

\emph{Response}: \textcolor{red}{Done.}

4 -- page 3, line 4, where it is written ``gamma density unit scale and shape \(a>0\),'' I suggest replacing it by ``gamma density with unit scale parameter and shape parameter \(a>0\),''.

\emph{Response}: \textcolor{red}{Done.}

5 -- page 3, \(3^{rd}\) paragraph, replace ``Table 2'' by ``Table 1''.

\emph{Response}: \textcolor{red}{Done.}

6 -- page 3, line-8, correct for ``(0,1)''.

\emph{Response}: \textcolor{red}{Done.}

7 -- Figure 1: I suggest to change the lines to different types (even letting the colours as they are) so as to enable the interpretation in a black and white print. Also, the letters assigned to the parameters must agree with those in Table 1 (and, eventually, with those in expression (3.1)).

\emph{Response}:
\textcolor{red}{Thanks a lot for the suggestion. All changes have been made. In fact, in a black and white print, solid lines will not help to differentiate the graphics.}

8 -- page 3, lines-8,-7, where it is written ``of the baseline \(Q_F\) \(Q_G(.)\).'' I suggest replacing it by ``of the \(Q_F\) of the baseline distribution \(G\), \(Q_G(.).\)''.

\emph{Response}: \textcolor{red}{Done.}

9 -- page 3, line-6, correct for ``(1.1) and''.

\emph{Response}:

10 -- page 7: the label of the table should be on the top, like the rest of the tables.

\emph{Response}: \textcolor{red}{Done.}

11 -- page 11, line 4, correct for ``distribution''.

\emph{Response}: \textcolor{red}{Done.}

12 -- In expressions (6.4) and (6.6), try to improve the output as the reference numbers are to close to the formulas.

\emph{Response}: \textcolor{red}{Done.}

13 -- I suggest to make the labels of Figure 1 and of tables in \LaTeX, not in R.

\emph{Response}:
\textcolor{red}{Thanks for the suggestion. The change suggestion has been implemented.}

14 -- The references ``R Core Team. \ldots{}'' and ``Rizzo, \ldots{}'' are in the wrong order.

\emph{Response}:

\textbf{REVIEW II}:

1 - The article is written in an overly summarized form and is sometimes unattractive to the reader of any journal. That alone could lead to your rejection. In addition, I do suggest reviewing the use of English carefully, and considerable rewording and pruning to make the paper more concise and precise.

\emph{Response}:

\textcolor{red}{We have improved the English language of the article in all sections.}

2 - When I see a new paper about a new distribution, I wonder if it is due to any interesting properties or it arises naturally in some
observable process like natural phenomena. I would like to see these
arguments defended more clearly in the article by the authors. Could you do that?



\emph{Response}:

\textcolor{red}{The Marshall and Olkin-G and Gamma-G generators, each with one more parameter, are often used to build new distributions. We empirically show through two real datasets that by combining these two generators, we obtain a new family of distributions that is better than two families widely used in the literature: beta and Kumarawamy (each one of the three with two parameters). In fact, the ``beta-G family of distributions" has over 3100 citations on google, whereas the ``Kumaraswamy-G family of distributions" has over 4500 citations  up to this date (May, 2022). Further, the properties of the proposed family are obtained similarly from the other two families using exponentiated distributions.} 

3 - The bibliography does not have many recent publications such as
Martinez-Florez et al. (2020) and Magalhaes et al.~(2020). I also
suggest authors to create the state-of-art related to this topic in the first section that definitively needs to be improved.

\emph{Response}:

4 - I expect to see more detailed discussion in the simulation study and real data analysis. Could you revise these sections, including the conclusion section that is actually weak?

5 - Suggestions: (i) Rewrite the paper title since the current one is nothing enlightening. How about naming the proposed distribution?. (ii) Place section 6 (properties) before section 4 (simulation). (iii) Reorganize the presentation of the tables. (iv) Put the simulation code in Supplementary material.

\emph{Response}:

\begin{itemize}
\item
  \begin{enumerate}
  \def\labelenumi{(\roman{enumi})}
  \tightlist
  \item
    \textcolor{red}{ Really, the title of the paper was no good. We changed to \textbf{Marshall and Olkin-Gamma-G family of distributions: properties and applications}. We believe that the title now clarifies the paper's proposal better. It is worth mentioning that the new name is consistent with the sequential application of the MO family to the baseline Gamma-G.}
  \end{enumerate}
\item
  \begin{enumerate}
  \def\labelenumi{(\roman{enumi})}
  \setcounter{enumi}{1}
  \tightlist
  \item
    \textcolor{red}{Done. Thanks for the suggestion.}
  \end{enumerate}
\item
  \begin{enumerate}
  \def\labelenumi{(\roman{enumi})}
  \setcounter{enumi}{2}
  \tightlist
  \item
    \textcolor{red}{Thanks for this point. The presentation of the tables was reorganized.}
  \end{enumerate}
\item
  \begin{enumerate}
  \def\labelenumi{(\roman{enumi})}
  \setcounter{enumi}{3}
  \tightlist
  \item
    \textcolor{red}{Thanks for the suggestion. We chose to put the code for the simulations in a repository on GitHub. Thus, we considerably reduced the number of pages in the article and made it easier for readers of the material to reuse the code. With the code on GitHub it is also possible to verify the implementation and validate the simulations obtained. To access the codes, see the link 
    \url{https://github.com/prdm0/MOGG/tree/master/code/simulations}}.
  \end{enumerate}
\end{itemize}

\closing{Yours Sincerely,}




 \end{letter}

\end{document}
